\documentclass[11pt]{article}
\usepackage[a4paper, total={15cm, 20cm}]{geometry}
\usepackage{graphicx}
\usepackage{array}
\usepackage{setspace}
\setstretch{1.5}

\usepackage[svgnames]{xcolor}
\usepackage{lettrine}
\usepackage[T1]{fontenc}
\usepackage{ebgaramond}
\renewcommand{\LettrineFontHook}{\color{Red}\initials}

%hyperlinks
\usepackage{hyperref}
\hypersetup{%
  pdftitle={Parallels between late-stage capitalism and the cyberpunk work Neuromancer},
  pdfauthor={490477027},%
  colorlinks = true,%
  linkcolor = red,%
  anchorcolor = red,%
  citecolor = red,%
  urlcolor = red%
}

%macro for italicized "Neuromancer"
\newcommand\neuro{\textit{Neuromancer}}

%macro for adding section to table of contents
\newcommand\addtotocsection[1]{\addcontentsline{toc}{section}{#1}}
\newcommand\addtotocsubsection[1]{\addcontentsline{toc}{subsection}{#1}}
\newcommand\addtotocsubsubsection[1]{\addcontentsline{toc}{subsubsection}{#1}}

%macros for footnote references
\newcommand\hasslerref{D.M. Hassler, \textit{ New Boundaries in Political Science Fiction.} (2008)}
\newcommand\bethkeref{B. Bethke, \textit{Cyberpunk: a short story by Bruce Bethke.} (1997)}
\newcommand\mandelref{E. Mandel, \textit{Late Capitalism.} (1972)}
\newcommand\hardtref{M. Hardt \& K. Weeks, \textit{The Jameson Reader.} (2000)}
\newcommand\lowreyref{A. Lowrey, \textit{Why the Phrase `Late Capitalism' Is Suddenly Everywhere.} (2017)}
\newcommand\kennonref{J. Kennon, \textit{What Are Commodities and How Do You Trade Them?} (2018)}
\newcommand\neuroref{M. Gibson, \textit{Neuromancer.} (1984)}
\newcommand\marxref{K. Marx, ``Estranged Labour'' \textit{Economic \& Philosophic Manuscripts of 1844.} (1844)}
\newcommand\hansonref{D. Hanson, \textit{``Gattungswesen'' – Species-being/Species Essence.} (2014)}
\newcommand\coxref{J. Cox, \textit{AN INTRODUCTION TO MARX'S THEORY OF ALIENATION.} (1998)}

\begin{document}
	\begin{titlepage}
		\begin{center}
			%\vspace{1cm}
			\textbf{\Huge Parallels between late-stage capitalism and the cyberpunk work \textit{Neuromancer}}\\
			%\vspace{1cm}
			\Large\textit{How do the characters and setting of the cyberpunk work \emph{Neuromancer} parallel the conditions and culture of late-stage capitalism?}\\
			\vspace{2cm}
			\Large Extended Essay - Literature\\
			\vspace{1cm}
			\textbf{\Large 490477027}\\
			\vspace{1cm}
			\large Work recompiled for the \LaTeX\ OLET1666 assignment\\
			\vspace{1cm}
			\includegraphics{usydlogo}\\
			\vspace{1cm}
			\large University of Sydney\\
			\large OLET 1666\\
			\large 24 August 2020\\
		\end{center}
	\end{titlepage}
	
	\tableofcontents
	
	\newpage
	
	\section*{Cyberpunk}\addtotocsection{Cyberpunk}
	\textsc{Cyberpunk} has always set itself apart as a unique genre in the wave of science fiction writing since the birth of sci-fi in the 1960’s. Unlike other future-oriented works of sci-fi, cyberpunk often portrays the near future of society rather than a world hundreds of years from now. Values, technology and society, are evolved and weaved into the future, resulting in settings that are simultaneously familiar and strange. A defining characteristic of the genre is the frequent juxtaposition of advanced technological and scientific achievements with a degree of breakdown or radical change in the social order\footnote{\hasslerref: p. 75-76}.
	
	\section*{Neuromancer}\addtotocsection{Neuromancer}
	\textsc{The} novel \neuro, by William Gibson, is generally regarded as the definitive benchmark of a cyberpunk novel. However – admittedly strangely for a literature essay – the focus of this essay lies not in the narrative of the novel. Indeed, a defining characteristic of \neuro\ as well as its (arguably) most meaningful addition to literature is in the vision put forth of the future. What Gibson has presented is a compelling vision of a possible future humanity is bound for. Increased dependence on technology, detachment amongst individuals, and a blurring of economic borders between nations, all eventually converging upon that iconic central vision of cyberspace. It is this prescient vision that elevated \neuro\ to its status as one of the most influential works of science fiction and that forged the beginning of the cyberpunk genre. Even today, \neuro\ still stands as the quintessential cyberpunk novel; it simply is cyberpunk. To this end, author Mike Swanwick once argued that writers of the cyberpunk movement should properly be termed ``neuromantics'' since so much of what they were doing was clearly imitation \neuro\footnote{\bethkeref}. Its enduring quality remains in Gibson’s prophetic vision of an ever-approaching future, one which has been transformed by technologies yet unknown. Furthermore, a striking feature of Neuromancer is that its world as presented by Gibson echoes aspects of Marxist theories regarding late-stage capitalism. Exploration of these parallels is the focus of this essay.
	
	\section*{Late-stage capitalism}\addtotocsection{Late-stage capitalism}
	\textsc{The} term `late-stage capitalism' was popularized by the Marxist economist Ernest Mandel in his 1972 PhD dissertation. The definition of the phrase stems from his theory of the three stages of the capitalist mode of production. Here, the capitalist mode of production refers to the systems of organizing production and distribution within capitalist societies, and is characterized by:
	\begin{enumerate}
		\item Private ownership of the means of production,
		\item Extraction of surplus value by the owning class for the purpose of capital accumulation,
		\item Wage-based labour, and
		\item Market-based appraisal of commodities\footnote{\mandelref}.
	\end{enumerate}
	Free competitive capitalism, an economic circumstance made possible by the Industrial Revolution, occurred from 1700-1850 and was characterised by the growth of industrial capital in domestic markets. Monopoly capitalism was characterized by the imperialistic development of international markets, including exploitation of colonial territories, lasting until approximately 1940. Finally, ``late capitalism'', emerging from the Second World War, has the dominant features of multinational corporations, globalised markets and labour, mass consumption, and multinational flow of capital\footnote{\hardtref: p. 165-166}. The term ``late-capitalism'' is generally used to refer to capitalism from about 1945 onwards, carrying the implication in the word \textit{`late'} that capitalism as a whole is due to come to an end\footnote{\lowreyref}.\\
	\smallbreak\noindent
	One of Mandel's conjectures of late-stage capitalism was that such a society would be dominated by the machinations of financial capital\footnote{\hardtref: p. 268-269}. In simpler terms, Mandel believed that a society of late-stage capitalism would be dominated by schemes to make money. It is reasonable to assume that Mandel, being a Marxist economist, implies the enactors of this to be multinational corporations, which is what Marxist theory believes to be problematic. Mandel was also insistent that late-stage capitalism would not signify the end of industrialization, stating:
	\begin{quote}
		\lettrine[lines=2,depth=3]{F}{ar}\textsc{ from representing a `post-industrial society', late capitalism thus constitutes \textbf{generalized universal industrialization} for the first time in history.}
		\flushright{--\mandelref: p. 387}
	\end{quote}
	According to him, these conditions of late-stage capitalism would result in the increasing commodification and industrialization of ever-more inclusive sectors of human life\footnote{\hardtref: p. 268-269}.\\
	\smallbreak\noindent
	It is from this premise of capitalism that parallels can be drawn to the cyberpunk work \neuro. The terms used to establish this premise will be defined prior to analysis. The two key terms are\break \textbf{commodification} and \textbf{industrialization}.
	
	\subsection*{Commodification}\addtotocsubsection{Commodification}
	\textsc{Commodification} is defined as:
	\begin{quote}
		\textsc{The action or process of treating something as a mere commodity.}
		\flushright{--``Commodification'', \textit{Oxford Dictionary}}
	\end{quote}
	In an economic sense, the term \textit{``commodity''} is used specifically for goods or services whose individual units are more or less interchangeable\footnote{\kennonref}. As previously established, Mandel believed that the conditions of late-stage capitalism would result in increasing commodification of ever-more inclusive sectors of human life. By taking this concept to an extreme, one can arrive at the commodification of human life itself. Here, the phrase \textit{``commodification of human life''} would refer to the treatment of a human life as a tradeable, interchangeable unit. This hypothetical is significant when combined with the knowledge that commodities in a capitalist mode of production are produced for the purpose of exchange and circulation in the market, with the ultimate end of obtaining a net profit. A depiction of this can be found in \neuro\ through the character Dixie Flatline.\\
	\smallbreak\noindent
	In the novel, Dixie Flatline is the moniker of the late legendary computer hacker McCoy Pauley. Before suffering a fatal heart attack, McCoy allowed the megacorporation Sense/Net to make a copy of his personality. As referenced in dialogue between two characters:
	\begin{quote}
		\begin{tabular}{>{\bf}r<{:}>{\it}p{10cm}}
			Case & Somebody’s got a recording of McCoy Pauley? Who?\\
			Molly & Sense/Net. Paid him mega, you bet your ass.\\
		\end{tabular}
		\flushright{--\neuroref: p. 57}
	\end{quote}
	In the world of \neuro, a person’s consciousness can be saved onto a ROM (read-only memory) cassette, thereby known as a `construct'. Said construct can be run as an artificial intelligence program that replicates the personality of the saved consciousness. As Case describes it:
	\begin{quote}
		\textit{It was disturbing to think of the Flatline as a construct, a hardwired ROM cassette replicating a dead man’s skills, obsessions, knee-jerk responses…}
		\flushright{--\neuro: p. 86}
	\end{quote}
	All interactions with this character that take place during the events of \neuro\ happen between the personality construct and others. In the novel, the protagonists needed the assistance of the legendary hacker, but at the time of the novel, McCoy had already died. However, his skills were still available in the form of the ROM construct of his consciousness. Sometime before the events of \neuro, the construct was purchased by the transnational megacorporation Sense/Net and was stored in the corporation’s data library, presumably for later use. As the previous dialogue states, McCoy sold the copy of his consciousness for a large sum of money. The construct is then acquired by breaking into the data library of Sense/Net, where the physical ROM cassette is stored. With the construct in the hands of the protagonists, the skills of the late McCoy Pauley falls into their possession, and his expertise is used to their advantage. The widened technological constraints in the world of \neuro\ allow the abstract concept of the human subconscious to be converted to a tangible object. Fundamentally, this is a reduction of the human essence to a base commodity. Pursuant to the capitalist mode of production, this commodity was purchased with the intent to create a gain in capital. Using the character of McCoy Pauley, the novel creates a representation of an aspect of the theory of late-stage capitalism that also considers the possible impacts of technological advancement.\\
	\smallbreak\noindent
	The character of McCoy ``Dixie Flatline'' Pauley – or more accurately, his construct – can thus be viewed as a representation of an extreme of the concept of commodification; that is, a human life reduced to a physical commodity. However, he is not the only such representation within \neuro. Another manifestation of the concept of human commodification can be found in the character of Hideo.\\
	\smallbreak\noindent
	Described as a ``vat-grown'' (i.e. genetically engineered) assassin, Hideo is employed as security by the wealthy transnational corporation Tessier-Ashpool. Passing dialogue illustrates the nature of his ``employment'':
	\begin{quote}
		\begin{tabular}{>{\bf}r<{:}>{\it}p{10cm}}
			Molly & You figure they \emph{(Tessier-Ashpool)} own that ninja, Finn?\\
			Finn & Smith thought so.\\
			Molly & Expensive. Wonder whatever happened to that little ninja, Finn?\\
			Finn & Probably got him on ice. Thaw when needed.\\
		\end{tabular}
		\flushright{--\neuro: p. 86}
	\end{quote}
	From the dialogue it can be seen that Hideo is thought of, not as a human being, but as no more than an object, used as a means to an end. Extending beyond mere servitude, the character is described to be ``vat-grown''. While the novel does not explicate the definition of ``vat-grown'', it can be inferred that it would entail the artificial construction of Hideo for the purpose he serves, meaning that a process of production was required in ``creating'' Hideo. Similarly to McCoy Pauley’s construct, Hideo is shown to be owned by a transnational corporation; in this case Tessier-Ashpool. The novel also references the existence of other ``vat-grown'' assassins twice in exposition:
	\begin{quote}
		\begin{tabular}{>{\bf}r<{:}>{\it}p{10cm}}
			Molly & You know, I figure the [assassin] Tessier-Ashpool sent after that Jimmy, the boy who stole the head, he must be pretty much the same as the one the Yak sent to kill Johnny.\\
		\end{tabular}
		\flushright{--\neuro: p. 197}
	\end{quote}
	\begin{quote}
		\begin{tabular}{>{\bf}r<{:}>{\it}p{10cm}}
			Molly & Because they’re \emph{(vat-grown assassins)} the best. Because one of them killed a partner of mine, once.\\
		\end{tabular}
		\flushright{--\neuro: p. 241}
	\end{quote}
	This reveals that the artificial ``production'' of a human for a specific task is an occurrence within the world of \neuro\ not unique to Hideo's character. From this it can be inferred that assassins of this type are purchased and used as objects; means to an end.\\
	\smallbreak\noindent
	There are many similarities between the character of Hideo and McCoy Pauley’s personality construct, the only difference being that Hideo – and other ``vat-grown'' assassins – still exist in a human body. However, both Hideo and McCoy Pauley’s construct are objectified in the same manner. Accordingly, the character of Hideo can also be seen as a representation of the concept of human commodification. Thus, it can be seen that these two characters within \neuro\ exhibit characteristics of the concept of commodification of human life. This is a clear parallel between the set and setting of the novel and conceptual theories of late-stage capitalism.
	
	\subsection*{Industrialization}\addtotocsubsection{Industrialization}
	\textsc{Another} way in which \neuro\ resembles late-stage capitalism is in the area of industrialization. The word is defined:
	\begin{quote}
		\textsc{The development of industries in a country or region on a wide scale.}
		\flushright{--``Industrialization'', \textit{Oxford Dictionary}}
	\end{quote}
	The concept of industrialization in \neuro\ is again seen through the lens of future technology and its hypothetical impacts. While industry in its most basic sense still exists and is incredibly prevalent, there are also more abstract notions of industrialization that exist in theories of late-stage capitalism. While this essay will deal with the fundamental concept of industry, it will also explore two more abstract theories of industrialization, being Mandel's hypothesis of \textsc{\textbf{Generalized Universal Industrialization}}, and Marx's theory of \textsc{\textbf{Labour Alienation}}, and how these theories can be seen in \neuro.
	
	\subsubsection*{Industry}\addtotocsubsubsection{Industry}
	\textsc{To} begin with, the basic concept of industry as well as how it pertains to Marxism must be defined, as it is from this core concept that the two abstract theories are derived. Again, using a dictionary definition first:
	\begin{quote}
		\textsc{Economic activity concerned with the processing of raw materials and manufacture of goods in factories.}
		\flushright{--``Industry'', \textit{Oxford Dictionary}}
	\end{quote}
	Material processing and factory-based manufacture of goods is industry at its most elementary, and it is the form of industry that theorists of the past such as Marx and Mandel would have concerned themselves about. This form of industry is also known to adversely affect the environment, seen in real life as a consequence of the Industrial Revolution. \neuro\ hints at this type of industrialization through passing descriptors of the general ecological collapse present. For example:
	\begin{quote}
		\textit{Case watched the sun ride on the landscape of childhood, on broken slag and the rusting shells of refineries.}
		\flushright{--\neuro: p. 95}
	\end{quote}
	In addition to direct depictions, Gibson also expresses the ecological collapse of \neuro\ indirectly. Along with outright descriptions of the ruined state of the environment, characters in the world of \neuro\ demonstrate an innate aversion to natural environment.\\
	\smallbreak\noindent
	The best example of this is in the memorable opening line of the novel:
	\begin{quote}
		\textit{The sky above the port was the colour of television, tuned to a dead channel.}
		\flushright{--\neuro: p. 1}
	\end{quote}
	Taken at surface level, the metaphor in this line only illustrates the colour of the sky. However, Gibson's choice of words establishes the prevalence of technology in the world of \neuro\ by describing a thing so distinctly organic and primitive such as the sky in terms of the unnatural and synthetic. This aversion to nature returns in a scene where the protagonist, Case, finds himself on a beach:
	\begin{quote}
		\textit{There seemed to be a city, beyond the curve of beach, but it was far away. […] The sky was a different silver. […] Tokyo Bay? He turned his head and stared out to sea, longing for the hologram logo of Fuji Electric, for the drone of a helicopter, anything at all.}
		\flushright{--\neuro: p. 257-258}
	\end{quote}
	Case is shown to feel unease when immersed in true nature, looking for comfort and familiarity in the synthetic. The quote is a subtle affirmation of the aversion to nature present within the world of \neuro. In this way Gibson expresses subtly the influence of industrialization implied to have taken place in the past.
	
	\subsubsection*{Generalized Universal Industrialization}\addtotocsubsubsection{Generalized Universal Industrialization}
	\textsc{A component} of Mandel’s description of the conditions of late capitalism is his theory of \textbf{\textsc{Generalized Universal Industrialization}}. He defines this as:
	\begin{quote}
		\textsc{Mechanization, standardization, overspecialization, and parcellization of labour…penetrat[ing] into all sectors of social life.}
		\flushright{--\mandelref: p. 387}
	\end{quote}
	This essay will focus on two of the four terms mentioned: \textbf{mechanization} and \textbf{overspecialization}.
	
	\subsubsection*{Mechanization and Overspecialization}\addtotocsubsubsection{Mechanization and Overspecialization}
	\textsc{Before} analysis can begin, the criteria of the terms should be defined first. If used specifically to refer to labour, \textbf{mechanization} is defined:
	\begin{quote}
		\textsc{Introduction of machines or automated devices into a process/place.}
		\flushright{--``Mechanize'', \textit{Oxford Dictionary}}
	\end{quote}
	\textbf{Overspecialization} is defined as:
	\begin{quote}
		\textsc{Concentrating too much on one aspect or area of something.}
		\flushright{--``Overspecialize'', \textit{Oxford Dictionary}}
	\end{quote}
	As the less abstract concept, mechanization is clearly evident within the world of \neuro. In the simplest sense of the word, any mention of technology is an allusion to mechanization. Overspecialization is a more ambiguous concept. However, this essay will make the point that the concept of overspecialization can be seen  manifested in the recurring motif of body modifications. Necessarily, body modifications being or utilizing a form of technology, they can also be seen to manifest the concept of mechanization.\\
	\smallbreak\noindent
	The rationale for the link between overspecialization and body modification is as follows: Overspecialization as it exists in the present world often manifests as the concentration of too much study on one area of knowledge. A conceptual extreme of this notion would be to modify one’s body such that it may perform a specific function. The modification of one’s body for a specific function can thus be seen as a physical manifestation of the notion of overspecialization. In this manner, the link between overspecialization and body modification has been justified.
	\smallbreak\noindent
	Of all the body modifications featured in the novel, the most striking belong to the character Molly Millions. Referred to as a \textsc{street samurai} or \textsc{Razorgirl}, Molly is a physically tough but unimposing mercenary-for-hire. During the events of \neuro, Molly works alongside Case, acting as his bodyguard and muscle. A distinct characteristic of Molly is her implementation of various body modifications and augmentations. Most conspicuous are the vision-enhancing lenses that seal her eye sockets, giving her the impression of wearing mirrored sunglasses, as well as ten razor-sharp retractable claws under her nails. Unseen modifications include artificially heightened reflexes and sensory input for combat. In essence, her entire body has been modified to serve as a weapon. It can be inferred that this transformation arose from a desire to excel in her profession as hired muscle. Thus, Molly can be seen as the epitome of the concept of overspecialization; her modifications concentrate solely on one aspect. Additionally, there is a link to Mandel’s theory of \textsc{Generalized Universal Industrialization} as a whole, in that said aspect relates to the purpose of creating capital; her profession is the main way in which she earns money.
	\smallbreak\indent
	To summarise this point, parallels can be drawn between a theory of late-stage capitalism and \neuro\ as Molly, a character in the novel, has been shown to feature aspects of overspecialization and mechanization, both of which are key terms within Mandel’s hypothesis of \textsc{Generalized Universal Industrialization}.
	
	\subsection*{Marx's Theory of Alienation}\addtotocsubsection{Marx's Theory of Alienation}
	In addition to being an example of overspecialization, Molly also exemplifies aspects of Marx's theory of alienation. This theory of alienation arose from Marx's consideration of the implications the Industrial Revolution had on the nature of labour in a capitalist mode of production. He theorizes a supposed estrangement between human labourers and their \textbf{\textsc{Gattungswesen}} – species-essence – in a capitalist society. In simpler terms, Marx believed that the labour required in a capitalist mode of production (defined above) was unnatural, and would consequently result in the estrangement of a worker from their humanity. Within this theory Marx outlines four types of alienation\footnote{\marxref}:
	\begin{enumerate}
		\item Alienation of the worker from their product
		\item Alienation of the worker from the act of production
		\item Alienation of the worker from their \textsc{Gattungswesen} (human nature)
		\item Alienation of the worker from other workers
	\end{enumerate}
	This essay will focus on the alienation of the worker from their \textsc{Gattungswesen}, as well as the alienation of the worker from other workers.
	
	\subsubsection*{Alienation of the Worker from their \textsc{Gattungswesen}}\addtotocsubsubsection{Alienation of the Worker from their Gattungswesen}
	\textsc{Gattungswesen} is a word that translates more-or-less to ``human nature''\footnote{\hansonref}. What Marx believed constituted such human nature was the ability to consciously shape the world. Marx argued that workers in a capitalist mode of production are alienated from their true human potential, human beings by nature possessing a \textsc{plurality of interests} and the tendency to engage in activities that promote psychological well-being and mutual human survival\footnote{\marxref}. Under capitalism, labour is forced; coerced\footnote{\coxref}. Work – and the worker – therefore become estranged from human nature. In essence, this form of alienation argues that the labour required of workers in a capitalist society is unnatural.\\
	\smallbreak\noindent
	To take the notion of alienation from human nature in a literal sense, the example of body modifications can be used. This is why the character of Molly serves as a representation of this concept. A symbolic representation of the departure from humanity in Molly's character can be seen in the way she cries. To accommodate the inset lenses that completely seal her eyes, her tear ducts were surgically routed into her mouth. As a result, whenever she cries, she either spits or swallows the tears. As revealed in dialogue:
	\begin{quote}
		\begin{tabular}{>{\bf}r<{:}>{\it}p{10cm}}
			John Harness Ashpool & How do you cry, Molly? I see your eyes are walled away. I'm curious.\\
			Molly & I don't cry, much.\\
			Ashpool & But how would you cry, if someone made you cry?\\
			Molly & I spit. The ducts are routed back into my mouth.\\
		\end{tabular}
		\flushright{--\neuro: p. 203}
	\end{quote}
	From this exchange, it can be seen that her mechanization in service to the capitalist mode of production results in the removal of a signifier of emotion, symbolically representing a clear departure from humanity. Beyond Molly, the fundamental concept of body modification itself exemplifies alienation from the human-essence. The replacement of any organic part of one’s body necessarily results in the increasing departure from one’s natural state.
	\smallbreak\noindent
	Detachment from humanity is also exemplified in another ``post-human'' character; the Dixie Flatline: McCoy Pauley. As stated previously, the memories and instincts of the late McCoy Pauley lived on after his death, housed in a personality construct. The way any given construct works in the world of \neuro\ is by using its housed memories and instincts to estimate its expressions, thus approximating the personality of the copied consciousness. However, said memories and instincts never learn or grow since a construct is comprised entirely of Read-Only Memory (ROM). Due to this, a personality construct is incredibly predictable. As stated in dialogue:
	\begin{quote}
		\begin{tabular}{>{\bf}r<{:}>{\it}p{10cm}}
			Wintermute & [Dixie Flatline]'s a construct, just a buncha ROM, so he always does what [you] expect him to.\\
		\end{tabular}
		\flushright{--\neuro: p. 226}
	\end{quote}
	The construct merely imitates a human life. The construct of the Flatline even shows awareness of this:
	\begin{quote}
		\begin{tabular}{>{\bf}r<{:}>{\it}p{10cm}}
			Flatline & Me, I’m not human either, but I respond like one. See?\\
			Case & Wait a sec. Are you sentient, or not?\\
			Flatline & Well, it \emph{feels} like I am, kid, but I'm really just a bunch of ROM. It's one of them, ah, philosophical questions, I guess…\\
		\end{tabular}
		\flushright{--\neuro: p. 145}
	\end{quote}
	The use of such a construct is that it retains the skills and expertise of the copied individual. It can be inferred that the intent behind the purchase of the construct of Dixie Flatline by the megacorporation Sense/Net was so that the skills of the legendary hacker could be put to use in their favour. A personality construct thus constitutes a worker who is completely estranged from their human essence. Therefore, it can be seen that the personality construct of the late McCoy Pauley – and personality constructs in the universe of \neuro\ as a whole – manifests Marx’s theory of alienation of the worker from their \textsc{Gattungswesen}. Pursuant to the capitalist mode of production, these constructs are produced with the intent to obtain a favoured position in the market, directly or otherwise.
	
	\subsubsection*{Alienation of the Worker from their other Workers}\addtotocsubsubsection{Alienation of the Worker from their other Workers}
	\textsc{The} capitalist mode of production reduces the labour of a worker to a commodity that can be traded in the labour-market. Capitalism thus provokes social conflict by pitting worker against worker in a competition for the highest wages, thereby alienating themselves from each other and the understanding of a mutual economic benefit\footnote{\marxref}. This form of alienation can be seen in \neuro, but as a variation on this theory. Rather, the estrangement in the novel manifests as an unwillingness or possibly even inability to form deep relationships due to the nature of a profession. This point again concerns the character of Molly. In her exposition of her past, she mentions a past lover with whom she maintained a relationship with. Summarising the expositional dialogue:
	\begin{quote}
		\begin{tabular}{>{\bf}r<{:}>{\it}p{10cm}}
			Molly & Had me this boy once. […] Johnny, his name was. […] Had the Yak \emph{(Yakuza)} after him, night I met him, and I did for \emph{(killed)} their assassin. […] And after that, it was tight and sweet, […] I was real happy. […] We worked together. Partners. […] Yakuza, I guess, they still wanted Johnny's ass. `Cause I’d killed their man. […] And the Yak, they can afford to move so fucking slow, man, they’ll wait years and years. […] But we were living fat, Swiss orbital accounts and a crib full of toys and furniture. Takes the edge off your game.\\
		\end{tabular}
		\flushright{--\neuro: p. 194-195}
	\end{quote}
	After this, she describes the death of Johnny at the hands of a ``vat-grown'' Yakuza assassin. Following this event, Molly remarks:
	\begin{quote}
		\begin{tabular}{>{\bf}r<{:}>{\it}p{10cm}}
			Molly & Never much found anybody I gave a damn about, after that.\\
		\end{tabular}
		\flushright{--\neuro: p. 196}
	\end{quote}
	Henceforth, Molly is shown to be obsessed with ``maintaining her edge''. In this manner, the competition within \neuro, instead of being between worker and worker for the purpose of attaining higher wages, is between Molly’s continuing effort to ``maintain her edge'' and the relationships she is a part of. It is apparent that the nature of her profession as a ``street samurai'' requires this of her. To this end, near the conclusion of the novel, Molly abruptly departs from Case, with whom a budding relationship was forming. She leaves the following note in their hotel room:
	\begin{quote}
		\textit{HEY ITS OKAY BUT ITS TAKING THE EDGE OFF MY GAME, I PAID THE BILL ALREADY. ITS THE WAY IM WIRED I GUESS, WATCH YOUR ASS OKAY?\\XXX MOLLY
}
		\flushright{--\neuro: p. 293}
	\end{quote}
	This letter demonstrates the extent of Molly’s alienation from others, encapsulating her cynical approach to life. In ``maintaining her edge'' for the sake of her profession, she pushes away relationships between her and others. This is the form of worker-worker alienation – different but still unmistakeably derived from Marx’s original theory that \neuro\ features.
	
	\section*{Conclusion}\addtotocsection{Conclusion}
	\textsc{\neuro} presents a fictional yet compelling vision of a world that resembles tenets of late-stage capitalism. Gibson's novel results in a visceral communication of the possible impacts of future technology, a perspective not seen in classical Marxist theories. Manifestations of abstract concepts can be seen in the characters and setting of the novel, all presented alongside Gibson's vision of the future. We see a world where technology has radically transformed the human experience. Gibson shows that inevitably, technology will become an increasingly integral part of human life. Consequently, human nature struggles to endure. As time passes, the answer to the question of what it means to be human becomes increasingly murky.
	
\end{document}








